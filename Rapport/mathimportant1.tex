
\begin{dfn} {\bf Variationsafledt}\\
Lad $F : C^\infty ([0,1]^n, \mathbb{R}^m) \to \mathbb{R}$ være et funktional over vektorrummet $C^\infty([0,1]^n, \mathbb{R}^m)$ over $\mathbb{R}$. Den variationsafledte af $F$ for $\mathbf{f}\in C^\infty ([0,1]^n, \mathbb{R}^m)$ i retningen ${\boldsymbol \nu}\in C^\infty ([0,1]^n, \mathbb{R}^m)$ er
\begin{align*}
\delta F(\mathbf{f}, {\boldsymbol \nu}) = \lim_{\tau \to 0} \frac{F(\mathbf{f}+\tau{\boldsymbol \nu})-F(\mathbf{f})}{\tau} = \frac{d}{d\tau} F(\mathbf{f}+\tau {\boldsymbol \nu}) \Big|_{\tau = 0}
\end{align*}
\end{dfn}


\begin{dfn} {\bf Variation}\\
En variation ${\boldsymbol \nu}$ er en funktion ${\boldsymbol \nu} \in C^\infty ([0,1]^n, \mathbb{R}^m)$ som opfylder $\textnormal{supp}\, {\boldsymbol \nu} \subset (0,1)^n$ (afslutningen af mængden hvori ${\boldsymbol \nu}$ ikke er identisk med nulfunktionen er en delmængde af $ (0,1)^n$), dvs. ${\boldsymbol \nu}$ er ihvertfald nulvektoren langs randen af $[0,1]^n$.
\end{dfn}


\begin{dfn} {\bf Lagrangefunktional i $\mathbb{R}^3$}\\
Lagrangefunktionalet $L : C^\infty ([0,1], \mathbb{R}^3) \to \mathbb{R}$ er defineret som integralet af en Lagrangefunktion $\mathcal{L} : [0,1] \times \mathbb{R}^3 \times \mathbb{R}^3 \to \mathbb{R}$ komposeret med $g=(t,{\boldsymbol \gamma}(t),\dot{\boldsymbol \gamma}(t))$ hvor ${\boldsymbol \gamma} \in C^\infty ([0,1], \mathbb{R}^3)$
\begin{align*}
L({\boldsymbol \gamma})= \int_0^1 (\mathcal{L}\circ g)(t) \, dt =\int_0^1 \mathcal{L}(t, {\boldsymbol \gamma}(t),\dot{{\boldsymbol \gamma}}(t)) \, dt
\end{align*}
\end{dfn}

\begin{dfn} {\bf Geodætisk Kurve}\\
En geodætisk kurve er en funktion ${\boldsymbol \gamma}_0 \in C^\infty ([0,1],\mathbb{R}^m)$ som opfylder $L({\boldsymbol \gamma}_0+\tau {\boldsymbol \nu}) \geq L({\boldsymbol \gamma}_0)$ for $\tau$ i et åbent interval om $0$ og for alle variationer ${\boldsymbol \nu}$ hvor funktionalet $L$ er afstanden $L({\boldsymbol \gamma})=\int^1_0 ||\dot{\boldsymbol \gamma}_0(t)|| \, dt$ (Lagrangefunktionen er normen $||\cdot||$).
\end{dfn}


\begin{thm} {\bf Fundamentalsætningen i Variationsregning}\\
Lad $f_m : [0,1]^n \to \mathbb{R}$ være en kontinuerlig funktion for hvert $m \in \{1, 2, \cdots, M \}$ og lad ligeledes $\nu_m : [0,1]^n \to \mathbb{R}$ være vilkårlige kontinuerlige funktioner $\textnormal{supp}\, \nu_m \subset (0,1)^n$.
\begin{align*}
\int_{[0,1]^n} \sum_{m=1}^M f_m \nu_m = 0 \quad \textnormal{for ethvert valg af $\nu_m$}
\end{align*}
hvis og kun hvis 
\begin{align*}
\forall m : f_m \equiv 0
\end{align*}
\end{thm}


\begin{proof} Reductio ad absurdum. Lad for hvert $m$ funktionerne $\nu_m $ være $\nu_m = f_m \eta_m$ hvor $\eta_m$ er endnu en vilkårlig funktion som opfylder betingelserne med den ekstra egenskab $\eta_m(x) >0$ i $\textnormal{supp}\, \eta_m$ således at $(f_m \nu_m)(x) = (f^2_m \eta_m)(x) \geq 0$ i hele $[0,1]^n$. Antag at for mindst en af funktionerne $f^2_p \eta_p$ et punkt $x_0 \in [0,1]^n$  eksisterer hvori $(f^2_p \eta_p)(x_0)>0$, idet funktionerne og dermed produktet er kontinuerligt, så eksisterer en åben kugle $B(x_0,\delta) \subset [0,1]^n$ om $x_0$ hvori $(f^2_p \eta_p)(x)>0$ (hvis ikke, så gælder for alle åbne kugler om $x_0$ at der eksisterer $x$ i kuglen så $(f^2_p \eta_p)(x)=0$ og $(f^2_p \eta_p)(x_0) \neq 0$ hvilket er i modstrid med kontinuitet). Integralet opfylder dermed $\int_{ B(x_0,\delta)} f^2_p \eta_p > 0$ og
\begin{align*}
0&=\int_{[0,1]^n} \sum_{m=1}^M f^2_m \eta_m  \\
&=\sum_{m=1}^M \Big(\int_{[0,1]^n\setminus B(x_0,\delta)}f^2_m \eta_m + \int_{ B(x_0,\delta)} f^2_m \eta_m \Big) \\
&\geq 0+\sum_{m=1}^M\int_{ B(x_0,\delta)} f^2_m \eta_m > 0
\end{align*}
hvilket er i modstrid med antagelsen og derfor $(f^2_m \eta_m)(x)= 0$ i $\textnormal{supp}\, \eta_m$ så $f_m(x) = 0$ i $\textnormal{supp}\, \eta_m$. Da $\eta_m$ var vilkårlig og $\textnormal{supp}\, \eta_m \subset (0,1)^n$ så $f_m(x) = 0$ i $(0,1)^n$ og sidst i $[0,1]^n$ som følge af kontinuitet langs randen - således følger $\forall m : f_m \equiv 0$. Omvendt hvis $\forall m : f_m \equiv 0$ så er integralet trivielt nul og sætningen er bevist.
\end{proof}


\begin{thm} {\bf Nødvendig betingelse for lokalt minimum}\\
Lad $F : C^\infty ([0,1]^n, \mathbb{R}^m) \to \mathbb{R}$. Hvis $F(\mathbf{f}_0+\tau {\boldsymbol \nu}) \geq F(\mathbf{f}_0)$ for alle $\tau\in (-\delta, \delta)$ og for alle variationer ${\boldsymbol \nu}$ så er $\delta F(\mathbf{f}_0, {\boldsymbol \nu})=0$ for alle variationer ${\boldsymbol \nu}$.
\end{thm}
\begin{proof} Bemærk at $F(\mathbf{f}_0+\tau {\boldsymbol \nu}) - F(\mathbf{f}_0) \geq 0$. Da den variationsafledte eksisterer (antages altid i denne rapport) så skal de to grænseværdier (i $(-\delta,0)$ mod højre og $(0,\delta)$ mod venstre)
\begin{align*}
\lim_{\tau \to 0^+} \frac{F(\mathbf{f}+\tau{\boldsymbol \nu})-F(\mathbf{f})}{\tau} \geq 0 \\
\lim_{\tau \to 0^-} \frac{F(\mathbf{f}+\tau{\boldsymbol \nu})-F(\mathbf{f})}{\tau} \leq 0
\end{align*}
være sammenfaldende $\delta F(\mathbf{f}_0, {\boldsymbol \nu})=0$ for alle variationer.
\end{proof}



\begin{thm} {\bf Euler-Lagrange Ligningerne i $\mathbb{R}^3$}\\
Lad $\mathcal{L} : [0,1] \times \mathbb{R}^3 \times \mathbb{R}^3 \to \mathbb{R}$ og ${\boldsymbol \gamma}_0 : [0,1] \to \mathbb{R}^3$ være $C^2$ funktioner. Førstevariationen opfylder $\delta L({\boldsymbol \gamma}_0, {\boldsymbol \nu}) = 0$ for alle variationer ${\boldsymbol \nu} :  [0,1] \to \mathbb{R}^3$ hvis og kun hvis
\begin{align}
\frac{\partial\mathcal{L}}{\partial x} -\frac{d}{dt}\frac{\partial\mathcal{L}}{\partial \dot{x}} &= 0 \\
\frac{\partial\mathcal{L}}{\partial y} -\frac{d}{dt} \frac{\partial\mathcal{L}}{\partial \dot{y}} &=0 \\
\frac{\partial\mathcal{L}}{\partial z} -\frac{d}{dt} \frac{\partial\mathcal{L}}{\partial \dot{z}} &=0
\end{align}
hvor $x$, $y$, $z$ refererer til anden, tredje og fjerde argumenter i $\mathcal{L}$ og lignende for $\dot{x}$, $\dot{y}$, $\dot{z}$.
\end{thm}
\begin{proof} I det følgende benyttes ombytning af integral og afledt (nemt at bevise tilladeligt for en $C^2$ integrand), partiel integration på hver komponent af typen $\frac{\partial\mathcal{L}}{\partial \dot{z}} \dot{\nu}_3$ og det faktum at komponenterne af $ {\boldsymbol \nu}$ opfylder $\nu_m(0)=0$ og $\nu_m(1)=0$ for $m=1,2,3$. Variationen er
\begin{align*}
\delta L({\boldsymbol \gamma}_0, {\boldsymbol \nu}) &= \frac{\partial}{\partial \epsilon} \int_0^1 \mathcal{L}(t, {\boldsymbol \gamma}_0(t)+\epsilon {\boldsymbol \nu}(t),\dot{{\boldsymbol \gamma}_0}(t)+\epsilon \dot{{\boldsymbol \nu}}(t)) \, dt \Big|_{\epsilon = 0} \\
&=
\int_0^1\frac{\partial}{\partial \epsilon} \mathcal{L}(t, {\boldsymbol \gamma}_0(t)+\epsilon {\boldsymbol \nu}(t),\dot{{\boldsymbol \gamma}_0}(t)+\epsilon \dot{{\boldsymbol \nu}}(t))\Big|_{\epsilon = 0} \, dt \\
&=
\int_0^1 \frac{\partial  \mathcal{\mathcal{L}}}{\partial t} \frac{\partial t}{\partial \epsilon} + \frac{\partial\mathcal{L}}{\partial x} \nu_1 + \frac{\partial\mathcal{L}}{\partial y} \nu_2 + \frac{\partial\mathcal{L}}{\partial z} \nu_3 + \frac{\partial\mathcal{L}}{\partial \dot{x}} \dot{\nu}_1 + \frac{\partial\mathcal{L}}{\partial \dot{y}} \dot{\nu}_2 + \frac{\partial\mathcal{L}}{\partial \dot{z}} \dot{\nu}_3 \, dt \\
&=
\int_0^1  \frac{\partial\mathcal{L}}{\partial x} \nu_1 + \frac{\partial\mathcal{L}}{\partial y} \nu_2 + \frac{\partial\mathcal{L}}{\partial z} \nu_3 -\frac{d}{dt}\frac{\partial\mathcal{L}}{\partial \dot{x}} \nu_1 -\frac{d}{dt} \frac{\partial\mathcal{L}}{\partial \dot{y}} \nu_2 -\frac{d}{dt} \frac{\partial\mathcal{L}}{\partial \dot{z}} \nu_3 \, dt \\
&=
\int_0^1 \Big( \frac{\partial\mathcal{L}}{\partial x} -\frac{d}{dt}\frac{\partial\mathcal{L}}{\partial \dot{x}} \Big) \nu_1  + \Big( \frac{\partial\mathcal{L}}{\partial y} -\frac{d}{dt} \frac{\partial\mathcal{L}}{\partial \dot{y}} \Big) \nu_2  +
\Big( \frac{\partial\mathcal{L}}{\partial z} -\frac{d}{dt} \frac{\partial\mathcal{L}}{\partial \dot{z}} \Big) \nu_3 \, dt \\
&=0
\end{align*}
hvis og kun hvis 
\begin{align*}
\frac{\partial\mathcal{L}}{\partial x} -\frac{d}{dt}\frac{\partial\mathcal{L}}{\partial \dot{x}} &= 0 \\
\frac{\partial\mathcal{L}}{\partial y} -\frac{d}{dt} \frac{\partial\mathcal{L}}{\partial \dot{y}} &=0 \\
\frac{\partial\mathcal{L}}{\partial z} -\frac{d}{dt} \frac{\partial\mathcal{L}}{\partial \dot{z}} &=0
\end{align*}
hvilket følger direkte af fundamentalsætningen i variationsregning.
\end{proof}
