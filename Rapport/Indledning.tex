\section{Indledning}
En geodæt er defineret som den kurve der realiserer den korteste strækning imellem 2 punkter. Er kurven ikke begrænset af en anden funktion, vil kurven være en ret linje mellem de to punkter. Er kurven derimod begrænset til at følge en funktion, f.eks. en rumkurve begrænset således at alle dens punkter er indeholdt i en flade, vil kurven kun kunne forbinde punkter der befinder sig på fladen, og hvis fladen ikke er et plan vil geodæten ikke nødvendigvis være en ret linje. Denne rapport søger at udlede ligningerne der bruges til at finde geodæter i planet og rummet, og derefter anvende disse ligninger til at finde geodæter for forskellige flader. Det vil for nogle flader være umuligt at finde analytiske løsninger til disse ligninger og vi vil derfor benytte CAS-programmet Maple til at løse ligningerne numerisk.